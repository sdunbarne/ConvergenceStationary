%%% -*-LaTeX-*-
%%% convergencestationary.tex.orig
%%% Prettyprinted by texpretty lex version 0.02 [21-May-2001]
%%% on Wed Feb  2 08:59:32 2022
%%% for Steven R. Dunbar (sdunbar@family-desktop)

%%% -*-LaTeX-*-
%%% convergencestationary.tex.orig
%%% Prettyprinted by texpretty lex version 0.02 [21-May-2001]
%%% on Mon Apr 26 09:12:53 2021
%%% for Steven R. Dunbar (sdunbar@family-desktop)

\documentclass[12pt]{article}

\input{../../../../etc/macros} %\input{../../../../etc/mzlatex_macros}
\input{../../../../etc/pdf_macros}

\bibliographystyle{plain}

\begin{document}

\myheader \mytitle

\hr

\sectiontitle{Convergence to the Stationary Distribution}

\hr

\usefirefox

\hr

% \visual{Study Tip}{../../../../CommonInformation/Lessons/studytip.png}
% \section*{Study Tip}

% \hr

\visual{Rating}{../../../../CommonInformation/Lessons/rating.png}
\section*{Rating} %one of
% Everyone: contains no mathematics.
% Student: contains scenes of mild algebra or calculus that may require guidance.
% Mathematically Mature: may contain mathematics beyond calculus with proofs.
Mathematicians Only:  prolonged scenes of intense rigor.

\hr

\visual{Section Starter Question}{../../../../CommonInformation/Lessons/question_mark.png}
\section*{Section Starter Question}

\hr

\visual{Key Concepts}{../../../../CommonInformation/Lessons/keyconcepts.png}
\section*{Key Concepts}

\begin{enumerate}
    \item
    \item
    \item
\end{enumerate}

\hr

\visual{Vocabulary}{../../../../CommonInformation/Lessons/vocabulary.png}
\section*{Vocabulary}
\begin{enumerate}
    \item
        Any vector \( (\pi)_{j} \) satisfying
        \begin{align}
            \sum\limits_{i} \pi_{i} P_{ij} &= \pi_{j},\\
            \sum\limits_{i} \pi_{i} &= 1
        \end{align}
        is a \defn{stationary probability distribution} of the Markov
        chain.
    \item
        The \defn{coupling inequality} is for two random variables \( X \)
        and \( Y \) each with its own distribution.  Then for any subset
        \( A \),
        \begin{multline*}
            \abs{\Prob{X \in A} - \Prob{Y \in A}} = \\
            \abs{\Prob{X \in A, X = Y} + \Prob{X \in A, X \ne Y } -
            \Prob{Y \in A, X = Y} - \Prob{Y \in A, X \ne Y}}.
        \end{multline*}
    \item
        A Markov chain with state space \( \mathcal{X} \) and transition
        probabilities \( P \) satisfies a \defn{minorization} condition
        if there exists a measurable subset \( C \subseteq \mathcal{X} \),
        a probability measure \( \mu \) on \( \mathcal{X} \), a constant
        \( \epsilon > 0 \), and a positive integer \( n_0 \), such that
        \[
            P^{n_0} (x, \cdot) \ge \epsilon \mu(\cdot), x \in C.
        \]
    \item
        Call a set \( C \) satisfying a minorization condition a \defn{small
        set}.
    \item
        If \( C = \mathcal{X} \), the entire state space, then the
        Markov chain satisfies a \defn{uniform minorization condition},
        also called \defn{Doeblin's condition}.
    \item
        The \defn{Hamming weight} of a vertex of the hypercube is the
        sum of the coordinates of the vertex,
    \item
        A Markov chain with a small set \( C \) satisfies a \defn{univariate
        drift condition} if there are constants \( 0 < \lambda < 1 \)
        and \( b < \infty \) and a function \( V :  \mathcal{X} \to [1,
        \infty] \) such that
        \[
            PV(x) \le \lambda V(x) + b \indicator{C}(x)
        \] where \( PV(x) = \E{V(X_{n+1}) \given X_n = x} \).
    \item
        A Markov chain with a small set \( C \subseteq X \) satisfies a
        \defn{bivariate drift condition} if there exists a function
        \[
            h :  \mathcal{X} \times \mathcal{X} \to [1, \infty)
        \] and some \( \alpha > 1 \) such that \( \bar{P}h(x,y) \le h(x,y)/\alpha
        \) for \( (x,y) \notin C \times C \) where
        \[
            \bar{P}h(x,y)= \E{h(X_{n+1}, Y_{n+1}) \given X_n = x, Y_n =
            y}
        \] is the expected value of \( h(X_{n+1}, Y_{n+1}) \) on the
        next iteration, when the chains start from \( x \) and \( y \)
        respectively.
\end{enumerate}

\hr

\visual{Mathematical Ideas}{../../../../CommonInformation/Lessons/mathematicalideas.png}
\section*{Mathematical Ideas}

Recall the Fundamental Theorem of Markov Chains:

\begin{theorem}[Fundamental Theorem of Markov Chains]
    For an irreducible, positive recurrent and aperiodic Markov chain \(
    \lim_{n \to \infty} (P^n)_{ij} \) exists and is independent of \( i \).
    Furthermore, letting
    \[
        \pi_j = \lim_{n \to \infty} (P^n)_{ij}
    \] then \( \pi_j \) is the unique non-negative solution of
    \begin{align}
        \sum\limits_{i} \pi_{i} P_{ij} &= \pi_{j},%
        \label{eq:convergencestationary:FTMC1}\\
        \sum\limits_{i} \pi_{i} &= 1.%
        \label{eq:convergencestationary:FTMC2}
    \end{align}
\end{theorem}
\index{Fundamental Theorem of Markov Chains}

\begin{definition}
    Any vector \( (\pi)_{j} \) satisfying equations~%
    \ref{eq:convergencestationary:FTMC1} and~%
    \ref{eq:convergencestationary:FTMC2} is a \defn{stationary
    probability distribution}%
    \index{stationary probability
    distribution}
    of the Markov chain.  A Markov chain started according to a
    stationary distribution \( (\pi_j) \) will have this distribution
    for all future times.
\end{definition}
\index{stationary probability distribution}

\begin{remark}
    This theorem says a probability transition matrix for an irreducible
    ergodic Markov chain has a \emph{left} eigenvector with
    corresponding eigenvalue \( 1 \).  This is a special case of the
    more general Perron-Frobenius Theorem.
\end{remark}

\begin{remark}
    The Fundamental Theorem says that under appropriate conditions, the
    powers of the probability transition matrix converge to the
    stationary distribution but gives no information about the rate of
    convergence.  The goal of this section is to prove some theorems and
    quantitative estimates about the rate of convergence.
\end{remark}
\index{convergence rate}

\subsection*{Convergence Theorem by Total Variation}

\begin{theorem}[Convergence Theorem by Total Variation]
    Suppose \( P \) is the transition probability matrix for an
    irreducible and aperiodic Markov chain with stationary distribution \(
    \pi \).  Then there exist constants \( \alpha \in (0,1) \) and \( C
    > 0 \) such that
    \[
        \max_{x \in \mathcal{X}} \| (P^n)_{i \cdot} - \pi \|_{TV} \le C
        \alpha^n.
    \]
\end{theorem}
\index{convergence!total variation}
\index{total variation!convergence}

\begin{proof}
    \begin{enumerate}
        \item
            Since \( P \) is irreducible and aperiodic, there exist \( r
            \) such that \( P^r \) has strictly positive entries.
        \item
            Let \( \Pi \) be the matrix with \( \mathcal{X} \) rows,
            each of which is the row vector \( \pi \).
        \item
            For sufficiently small \( \delta > 0 \), \( (P^r)_{ij} \ge
            \delta \pi_{j} \) for \( i,j \in \mathcal{X} \).
        \item
            Let \( \theta = 1-\delta \).  Define stochastic matrix \( Q \)
            by
            \[
                P^r = (1 - \theta)\Pi + \theta Q.
            \]
        \item
            Note that \( M \Pi = \Pi \) for any stochastic matrix and \(
            \Pi M =\Pi \) for any matrix with \( \pi M = \pi \).
        \item
            Claim:
            \[
                P^{rk} = (1- \theta^{k})\Pi + \theta^k Q^k.
            \] Proof by Induction:  For \( k=1 \), this is the
            definition of \( Q \).  Assume the claim is true for \( k=n \).
            \begin{align*}
                P^{r(k+1)} &= P^{rn} P^r = [(1-\theta^n)\Pi + \theta^n Q^n]P^r
                \\
                &= (1-\theta^n)\Pi P^r + (1-\theta) \theta^n Q^n \Pi +
                \theta^{n+1} Q^n Q.
            \end{align*}
            Use \( \Pi P^r = \Pi \) and \( Q^n \Pi = \Pi \).
            \[
                P^{r(k+1)} = (1- \theta^{k+1})\Pi + \theta^{k+1} Q^{k+1}.
            \] Hence the relation holds for all \( k \).
        \item
            Multiply by \( P^j \), and rearrange to obtain
            \[
                P^{rk+j} - \Pi = \theta^k (Q^k P^j - \Pi).
            \]
        \item
            Sum the absolute value of row \( x_0 \) on both sides and
            divide by \( 2 \).  On the right, the absolute row \( x_0 \)
            sum from \( (Q^k P^j - \Pi) \) is at most the largest
            possible total variation distance between distributions (which
            is at most \( 1 \)).  Hence
            \[
                \| (P^{rk+j})_{i \cdot} - \pi \|_{TV} \le \theta^k.
            \]
        \item
            To finish the proof, let \( \alpha = \theta^{1/r} \) and \(
            C= 1/\theta \).
    \end{enumerate}
\end{proof}

This theorem and is proof has the advantage of being relatively
straightforward depending on matrix multiplications.  However,it is not
easily descriptive of the rate of convergence.

\subsection*{Eigenvalue Analysis}

Consider the \( 3 \times 3 \) square lattice graph in Figure~%
\ref{fig:convergencestationary:sqlattice}.  The graph has \( 9 \)
vertices with \( 10 \) edges between nearest lattice neighbors.  If a
vertex has \( n \) edges then the probability of moving to a neighboring
edge or staying at the vertex is \( \frac{1}{n+1} \) uniformly.  The
random walk on this graph is often colorfully characterized as a frog
hopping among lily pads or a bug leaping among plants.  The transition
probability matrix for this random walk is
\[
    \begin{pmatrix}
        1/3 & 1/3 & 0 & 1/3 & 0 & 0 & 0 & 0 & 0 \\
        1/4 & 1/4 & 1/4 & 0 & 1/4 & 0 & 0 & 0 & 0 \\
        0 & 1/3 & 1/3 & 0 & 0 & 1/3 & 0 & 0 & 0 \\
        1/4 & 0 & 0 & 1/4 & 1/4 & 0 & 1/4 & 0 & 0 \\
        0 & 1/5 & 0 & 1/5 & 1/5 & 1/5 & 0 & 1/5 & 0 \\
        0 & 0 & 1/4 & 0 & 1/4 & 1/4 & 0 & 0 & 1/4 \\
        0 & 0 & 0 & 1/3 & 0 & 0 & 1/3 & 1/3 & 0 \\
        0 & 0 & 0 & 0 & 1/4 & 0 & 1/4 & 1/4 & 1/4 \\
        0 & 0 & 0 & 0 & 0 & 1/3 & 0 & 1/3 & 1/3
    \end{pmatrix}
    .
\]

This Markov chain has a stationary distribution
\[
    \pi = (\frac{1}{11}, \frac{4}{33}, \frac{1}{11}, \frac{4}{33}, \frac
    {5}{33}, \frac{4}{33}, \frac{1}{11}, \frac{4}{33}, \frac{1}{11}).
\] This Markov chain is ergodic and aperiodic.  All states are
accessible and all states communicate.  This Markov chain has no
transient states and all states are recurrent.  This Markov chain has no
absorbing states.

\begin{figure}
    \centering
\begin{asy}
size(5inches);

real myfontsize = 12;
real mylineskip = 1.2*myfontsize;
pen mypen = fontsize(myfontsize, mylineskip);
defaultpen(mypen);

real marge=1mm;
pair z1=(0, 2), z2=(1, 2), z3=(2, 2);
pair z4=(0, 1), z5=(1, 1), z6=(2, 1);
pair z7=(0, 0), z8=(1, 0), z9=(2, 0);

transform r=scale(1.0);

object state1=draw("1",ellipse,z1,marge),
state2=draw("2",ellipse,z2,marge),
state3=draw("3",ellipse,z3,marge),
state4=draw("4",ellipse,z4,marge),
state5=draw("5",ellipse,z5,marge),
state6=draw("6",ellipse,z6,marge),
state7=draw("7",ellipse,z7,marge),
state8=draw("8",ellipse,z8,marge),
state9=draw("9",ellipse,z9,marge);

add(new void(picture pic, transform t) {
    draw(pic, point(state1,E,t)--point(state2,W,t));
    draw(pic, point(state1,S,t)--point(state4,N,t));
});

add(new void(picture pic, transform t) {
    draw(pic, point(state2,E,t)--point(state3,W,t));
    draw(pic, point(state2,S,t)--point(state5,N,t));
});

add(new void(picture pic, transform t) {
    draw(pic, point(state3,S,t)--point(state6,N,t));
});

add(new void(picture pic, transform t) {
    draw(pic, point(state4,E,t)--point(state5,W,t));
    draw(pic, point(state4,S,t)--point(state7,N,t));
});

add(new void(picture pic, transform t) {
    draw(pic, point(state5,E,t)--point(state6,W,t));
    draw(pic, point(state5,S,t)--point(state8,N,t));
});

add(new void(picture pic, transform t) {
    draw(pic, point(state5,E,t)--point(state6,W,t));
    draw(pic, point(state5,S,t)--point(state8,N,t));
});

add(new void(picture pic, transform t) {
    draw(pic, point(state6,S,t)--point(state9,N,t));
});

add(new void(picture pic, transform t) {
    draw(pic, point(state7,E,t)--point(state8,W,t));
});

add(new void(picture pic, transform t) {
    draw(pic, point(state8,E,t)--point(state9,W,t));
});
\end{asy}
    \caption{A $3 \times 3$ square lattice graph with uniform
    transition probabilities.}%
    \label{fig:convergencestationary:sqlattice}
\end{figure}

The eigenvalues are
\begin{align*}
    \lambda_0 &= 1, \\
    \lambda_1 &= \frac{7 + \sqrt{97}}{24} \approx 0.702, \\
    \lambda_2 &= \frac{7 + \sqrt{97}}{24} \approx 0.702, \\
    \lambda_3 &= \frac{1}{3}, \\
    \lambda_4 &= \frac{1}{4}, \\
    \lambda_5 &= \frac{1}{4}, \\
    \lambda_6 &= \frac{7 - \sqrt{97}}{24} \approx -0.119, \\
    \lambda_7 &= \frac{7 - \sqrt{97}}{24} \approx -0.119, \\
    \lambda_8 &= -\frac{7}{15} \approx -0.467,
\end{align*}
The corresponding left eigenvectors are

Suppose the random walk starts from the center of the square lattice,
state \( 5 \), that is \( X_0 \sim (0,0,0,0,1,0,0,0,0) \).

\subsection*{Coupling and Minorization}

The idea of coupling is to create two different copies of a random
object, and compare them.  A key idea is the \defn{coupling inequality}.%
\index{coupling inequality}
Suppose \( X \) and \( Y \) are two random variables, each with its own
distribution.  Then for any subset \( A \),
\begin{multline*}
    \abs{\Prob{X \in A} - \Prob{Y \in A}} = \\
    \abs{ \Prob{X \in A, X = Y} + \Prob{X \in A, X \ne Y } - \Prob{Y \in
    A, X = Y} - \Prob{Y \in A, X \ne Y}}
\end{multline*}
However, \( \Prob{X \in A, X = Y} = \Prob{Y \in A, X = Y} \) since both
refer to the same event.  Also, \( 0 \le \Prob{X \in A, X \ne Y} \le
\Prob{X \ne Y} \) and \( 0 \le \Prob{Y \in A, X \ne Y} \le \Prob{X \ne Y}
\) so \( \Prob{X \in A, X \ne Y} - \Prob{Y \in A, X \ne Y} \le \Prob{X
\ne Y} \).  Hence
\[
    \abs{\Prob{X \in A} - \Prob{Y \in A}} \le \Prob{X \ne Y}.
\] Since the upper bound is uniform over \( A \) this gives a bound on
the total variation distance
\begin{equation}
    \label{eqn:convergencestationary:tvbound} \|
    \operatorname{dist}
    (X) =
    \operatorname{dist}
    (Y) \|_{TV} = \sup_{A \subseteq \mathcal{X}} \abs{\Prob{X \in A} -
    \Prob{Y \in A}} \le \Prob{X \ne Y}.
\end{equation}
\begin{definition}
    A Markov chain with state space \( \mathcal{X} \) and transition
    probabilities \( P \) satisfies a \defn{minorization condition}%
    \index{minorization condition}
    if there exists a measurable subset \( C \subseteq \mathcal{X} \), a
    probability measure \( \mu \) on \( \mathcal{X} \), a constant \(
    \epsilon > 0 \), and a positive integer \( n_0 \), such that
    \[
        P^{n_0} (x, \cdot) \ge \epsilon \mu(\cdot), x \in C.
    \] Call a set \( C \) satisfying a minorization condition a \defn{small
    set}.%
    \index{small set}
    In particular, if \( C = \mathcal{X} \), the entire state space,
    then the Markov chain satisfies a \defn{uniform minorization
    condition},%
    \index{uniform minorization condition}
    also called \defn{Doeblin's condition}.%
    \index{Doeblin's condition}
\end{definition}

The uniform minorization condition implies there exists a common overlap
of size \( \epsilon \) between all of the transition probabilities. This
allows a coupling construction of two different copies \( X_n \) and \(
X_n' \) of a Markov chain as follows.  Assume for now \( n_0 = 1 \).
\begin{enumerate}
    \item
        Choose \( X_0 \sim \mu_0(\cdot) \) and \( X_0' \sim \pi(\cdot) \)
        independently.
    \item
        If \( X_n = X_n' \), choose \( z \sim P(X, \cdot) \) and let \(
        X_{n+1}' = X_{n+1} = z \).  Now the chains are coupled and the
        two chains will remain equal forever.
    \item
        If \( X_n \ne X_n' \), flip a coin whose probability of heads is
        \( \epsilon \).  If the coin shows heads, choose \( z \sim \mu(\cdot)
        \), and let \( X_{n+1}' = X_{n+1} = z \).  Otherwise, update \(
        X_{n+1} \) and \( X_{n+1}' \) independently with probabilities
        given by
        \begin{align*}
            \Prob{X_{n+1} \in A} &= \frac{P(X_n, A) - \epsilon \mu(A)}{1-\epsilon}
            \\
            \Prob{X_{n+1}' \in A} &= \frac{P(X_n', A) - \epsilon \mu(A)}
            {1-\epsilon}.  \\
        \end{align*}
\end{enumerate}

The minorization condition guarantees that the residual probabilities
are nonnegative and are probability measures since
\[
    \frac{P(X_n, \mathcal{X}) - \epsilon \mu(\mathcal{X})}{1-\epsilon} =
    \frac{1-\epsilon}{1-\epsilon} = 1.
\]

\begin{lemma}
    \( P(X_{n+1} \in A \given X_n = x) = P(x,A) \) and \( P(X_{n+1}' \in
    A \given X_n' = x) = P(x,A) \) for any \( x \in \mathcal{X} \)
\end{lemma}

\begin{proof}
    If the two chains are unequal at time \( n \), then
    \begin{align*}
        \Prob{X_{n+1} \in A \given X_n = x} &= \Prob{X_{n+1} \in A,\text
        { Heads} \given X_n = x} + \Prob{X_{n+1} \in A,\text{ Tails}
        \given X_n = x} \\
        &= \Prob{\text Heads} \Prob{X_{n+1} \in A \given X_n = x, \text{%
        Heads}} + \Prob{X_{n+1} \in A \given X_n = x, \text{ Tails} } \\
        &= \epsilon \mu(A) + (1-\epsilon) \frac{\Prob{X_n, \mathcal{X} -
        \epsilon \mu(\mathcal{X})}}{1-\epsilon} \\
        &= P(x,A).
    \end{align*}
\end{proof}

\begin{theorem}[Minorization Estimate]
    If \( X_n \) is a Markov chain on \( \mathcal{X} \) with transition
    probabilities satisfying a uniform minorization condition for some \(
    \epsilon > 0 \), then for any positive integer \( n \) and any \( x
    \in \mathcal{X} \)
    \[
        \|
        \operatorname{dist}
        (X_n) - \pi \|_{TV} \le (1-\epsilon)^{\lfloor n/n_0 \rfloor}.
    \]
\end{theorem}
\index{minorization estimate}

\begin{proof}
    In \( n_0 > 1 \), use the construction above for the times \( n = 0,
    n_0, 2n_0, \dots \) with \( n+1 \) replaced by \( n + n_0 \) and
    with \( P \) replaced by \( P^{n_0} \).  Then fill in the
    intermediate state \( X_n \) for \( j n_0 < n < (j+1) n_0 \) from
    the appropriate conditional distribution given the
    already-constructed values of \( X_{j n_0} \) and \( X_{(j_+1)n_0} \).

    Since \( X_0' \sim \pi \) and \( \pi \) is stationary, then \( X_n'
    \sim \pi \) for all \( n \).  Every \( n_0 \) steps, the two chains
    have probability at least \( \epsilon \) of coupling because the
    coin comes up Heads.  So

    \( \Prob(X_n \ne X_n') \le (1-\epsilon)^{\lfloor n/n_0 \rfloor} \).
\end{proof}

Assume \( \mathcal{X} \) is finite and for some \( n_0 \) there is at
least one state \( j \in \mathcal{X} \) such that the \( j \)th column
of \( P^{n_0} > 0 \) for all \( x \in \mathcal{X} \).  Then set \(
\epsilon = \sum_{x \in \mathcal{X}} \min_{i \in \mathcal{X}} (P^{n_0})_{i,j}
> 0 \) and \( \mu(j) = \epsilon^{-1} (P^{n_0})_{i,j} \) so \( (P^{n_0})_
{i,j} \ge \epsilon \mu(j) \) for all \( i,j \in \mathcal{X} \).  That
is, an \( n_0 \)-step minorization condition is satisfied with \(
\epsilon \).

\begin{example}
    Consider the random walk on the \( 3 \times 3 \) square lattice. The
    transition probabilities do not satisfy a minorization condition
    since every column has some zeros.  However, in \( P^2 \), column
    has all positive values,
    \[
        \begin{pmatrix}
            \frac{5}{18} & \frac{7}{36} & \frac{1}{12} & \frac{7}{36} &
            \frac{1}{6} & 0 & \frac{1}{12} & 0 & 0\\
            \frac{7}{48} & \frac{67}{240} & \frac{7}{48} & \frac{2}{15}
            & \frac{9}{80} & \frac{2}{15} & 0 & \frac{1}{20} & 0\\
            \frac{1}{12} & \frac{7}{36} & \frac{5}{18} & 0 & \frac{1}{6}
            & \frac{7}{36} & 0 & 0 & \frac{1}{12}\\
            \frac{7}{48} & \frac{2}{15} & 0 & \frac{67}{240} & \frac{9}{80}
            & \frac{1}{20} & \frac{7}{48} & \frac{2}{15} & 0\\
            \frac{1}{10} & \frac{9}{100} & \frac{1}{10} & \frac{9}{100}
            & \frac{6}{25} & \frac{9}{100} & \frac{1}{10} & \frac{9}{100}
            & \frac{1}{10}\\
            0 & \frac{2}{15} & \frac{7}{48} & \frac{1}{20} & \frac{9}{80}
            & \frac{67}{240} & 0 & \frac{2}{15} & \frac{7}{48}\\
            \frac{1}{12} & 0 & 0 & \frac{7}{36} & \frac{1}{6} & 0 &
            \frac{5}{18} & \frac{7}{36} & \frac{1}{12}\\
            0 & \frac{1}{20} & 0 & \frac{2}{15} & \frac{9}{80} & \frac{2}
            {15} & \frac{7}{48} & \frac{67}{240} & \frac{7}{48}\\
            0 & 0 & \frac{1}{12} & 0 & \frac{1}{6} & \frac{7}{36} &
            \frac{1}{12} & \frac{7}{36} & \frac{5}{18}
        \end{pmatrix}
        .
    \] The walk starting in state \( 5 \) will always have at least a
    probability \( \frac{9}{80} \) that it will return to state \( 5 \)
    in two steps.  Take the minorization condition as
    \[
        \epsilon = \sum_{x \in \mathcal{X}} \min_{j \in \mathcal{X}} (P^2)_
        {i,j} = 0 + 0 + 0 + 0 + \frac{9}{80} + 0 + 0 + 0 + 0 = \frac{9}{80}
    \] and \( \mu(j) = \epsilon^{-1} (P^2)_{i,j} \).  By the Theorem
    with \( n_0 = 2 \) and \( \epsilon = \frac{9}{80} \),
    \[
        \|
        \operatorname{dist}
        (Xn) - \pi \|_{TV} \le (1-\frac{9}{80})^{\lfloor n/2 \rfloor} -
        \left( \frac{71}{80} \right)^{\lfloor n/2 \rfloor}.
    \] To get the distribution of the random walk within \( 0.01 \) of
    the stationary distribution requires \( 78 \) steps.  Compare this
    to the estimate of \( 6 \) steps using the eigenvalues of \( P \).
\end{example}

\subsection*{An Example of the Cut-Off Phenomenon}

% https://pages.uoregon.edu/dlevin/MARKOV/mcmt2e.pdf
% https://sites.tufts.edu/vrdi/files/2019/06/Markov-Chains-and-Mixing-Times.pdf

This subsection shows some sharper estimates of the rate of convergence
using coupling in specific cases.  Here the case of random walk on the
hypercube exhibits the \emph{cut-off} phenomenon.%
\index{cut-off phenomenon}

Let \( Q^n \), the \( n \)-dimensional hypercube graph, with vertices or
node set \( V(Q^n) = \set{x_0, x_1, \dots x_n} = \set{0,1}^n \) and edge
set \( E(Q^n) = \setof{(x,y)}{x, y \text{ differ in one coordinate}} \).%
\index{hypercube}
As a notational convenience, let \( \bar{0} = (0, dots, 0) \) denote the
origin and \( \bar{1} = (1, dots, 1) \) be the extreme vertex of the
hypercube.

A random walk on this graph is the sequence \( X_0, X_1, \dots, X_{n-1},
X_n, \dots \) where given \( X_{n-1} \), choose \( X_n \) uniformly at
random from the nodes adjacent to \( x_{n-1} \).%
\index{random walk!hypercube}
A practical way to implement this random walk is to choose a coordinate \(
j \in \set{1,2, \dots, n} \) uniformly at random and flip the bit at \(
j \) from \( 0 \) to \( 1 \) or from \( 1 \) to \( 0 \).  For example,
on the \( 3 \)-dimensional cube a walk at \( 011 \) will move to \( 111 \)
is position \( 1 \) is selected, to \( 001 \) if position \( 2 \) is
selected, and \( 010 \) if position \( 3 \) is selected. As \( n \to
\infty \), the distribution of \( X_n \) converges to the uniform
distribution \( \pi \) on \( V(Q^n) \).  (See the exercises.) This
random walk has the annoying disadvantage of being periodic with period
two, so instead consider the \emph{lazy random walk}%
\index{lazy random walk}
sequence \( X_0, X_1, \dots, X_{n-1}m X_n, \dots \) where given \( X_{n-1}
\), first choose to remain at \( X_{n-1} \) with probability \( \frac{1}
{2} \).  Alternatively, with probability \( \frac{1}{2} \) choose to
move to \( X_n \) selected uniformly at random from the nodes adjacent
to \( x_{n-1} \).  A practical way to implement this random walk is to
choose a coordinate \( j \in \set{1,2, \dots, n} \) uniformly at random
and replace the bit in that position with the value of a fair Bernoulli
random value. The lazy random walk also converges to the uniform
distribution \( \pi \) on \( V(Q^n) \).  (See the exercises.)

The intention here to understand the rate of convergence to the
stationary distribution in the Total Variation norm.  More precisely,
given fixed \( \epsilon \) what the smallest \( t = t(n) \) such that
\[
    \| \Prob{X_t} - \pi \|_{TV} < \epsilon
\] as a function \( n \).

The investigation proceeds by coupling.%
\index{coupling}
Start one random walk \( X \) at \( \bar{0} \), and start another random
walk \( Y \) ``far'' from \( \bar{0} \). Define \( \tau_y = \inf \setof{t}
{X_t = Y_t, Y_0 = y} \).  Then define a combined random walk \( X_t = Y_t
\) for \( t > \tau_y \).  Since neither random walk is dependent on
history, but only on the current state, the probability of states in the
future is the same for each walk.  Combining the law of total
probability with the triangle inequality
\begin{align*}
    & \| \Prob{X_t = \cdot} - \Prob{Y_t} \|_{TV} \le \\
    & \qquad \| \Prob{X_t = \cdot \given \tau_y \le t} - \Prob{Y_t} =
    \cdot \given \tau_y \le t \|_{TV} \Prob{\tau_y \le t} +\\
    & \qquad \| \Prob{X_t = \cdot \given \tau_y > t} - \Prob{Y_t} =
    \cdot \given \tau_y \le t \|_{TV} \Prob{\tau_y > t} & \qquad = \|
    \Prob{X_t = \cdot \given \tau_y > t} - \Prob{Y_t} = \cdot \given
    \tau_y \le t \|_{TV} \Prob{\tau_y > t} & \le \Prob{\tau_y > t}.
\end{align*}
Notice the first summand is \( 0 \)
\[
    \| 0 = \Prob{X_t = \cdot \given \tau_y \le t} - \Prob{Y_t} = \cdot
    \given \tau_y \le t \|_{TV} \Prob{\tau_y \le t}
\] and the last inequality follows because \( \| \cdot \|_{TV} \le 1 \).
This inequality is basically the Strong Markov Property. {Why is this
so, it seems that it ought to be the second summand.}

Recall
\begin{align*}
    \| \Prob{X_t = \cdot} - \pi \|_{TV} &\le \max_{y} \| \Prob{X_t =
    \cdot} - \Prob{Y_t \given Y_0 = y} \|_{TV} \\
    &\le \| \Prob{X_t = \cdot} - \Prob{Y_t \given Y = \bar{1}} \|_{TV}
    \\
    &\le \Prob{\tau_{\bar{1}}} > t
\end{align*}

Let the \defn{Hamming weight}%
\index{Hamming weight}
of a vertex be the sum of the coordinates of the vertex, ranging from
Hamming weight \( 0 \) at \( \bar{0} \) to \( n \) at \( \bar{1} \).  As
notation, use \(
\operatorname{ht}
(x) \) for the hieght of vertex \( x \).  Consider the schematic
characterization of the hypercube in Figure~%
\ref{fig:convergencestationary:hypercube}. The plan for the remander of
the proof is show that both walks get to the bulk of the hypercube in \(
\frac{n}{2} \log n \) steps, and then it takes an additional \( O(n) \)
steps for them to meet.

\begin{figure}
    \centering
\begin{asy}
settings.outformat = "pdf";

import graph;

size(5inches);

real myfontsize = 12;
real mylineskip = 1.2*myfontsize;
pen mypen = fontsize(myfontsize, mylineskip);
defaultpen(mypen);

real center = 7;
real halfwidth = 4;		// perfect square
real height = 2 * sqrt(halfwidth) + 2 * halfwidth;
real l = center - halfwidth;
real r = center + halfwidth;
real h1 = sqrt(halfwidth);
real h2 = halfwidth;
real height = h1 + h2 + h1;
real rr = r + 1;

real f1( real x ) { return sqrt( abs( x - center ) ); }
real f2( real x ) { return height - sqrt(abs( x - center ) );} 

draw( graph(f1, l, r) );
draw( (l, h1)--(l, h1+h2) );
draw( (r, h1)--(r, h1+h2) );
draw( graph(f2, l, r) );

draw( (l, h1)--(r,h1), dashed);
draw( (l, height/2)--(r,height/2), dashed);
draw( (l, h1+h2)--(r,h1+h2), dashed);

label("$\bar{0}$", (center, 0), S);
label("$\bar{1}$", (center, height), N);

yaxis(L="Hamming Weight", ymin=0, ymax = height, autorotate=false);
ytick(Label("$0$", (0,0), E), (0,0));
ytick(Label("$n/2 - O(\sqrt{n})$", (0,h1), SE), (0, h1));
ytick(Label("$n/2$", (0, height/2), E), (0,height/2));
ytick(Label("$n/2 + O(\sqrt{n})$", (0,h1+h2), NE), (0, h1+h2));
ytick(Label("$n$", (0,height), E), (0,height));

Label L1= Label("steps $\approx \frac{n}{2} \log n$", align=(0,0),
		position=MidPoint, filltype=Fill(white));
draw((rr,0)--(rr,h1), L=L1, bar=BeginBar, arrow=EndArrow);
draw((rr,height)--(rr,h1+h2), L=L1, bar=BeginBar, arrow=EndArrow);
\end{asy}
    \caption{Caricature of the hypercube in terms of the Hamming weight
    of the vertices.}%
    \label{fig:convergencestationary:hypercube}
\end{figure}

An algorithm to simulate \( X_t \) and \( Y_t \) simultaneously is the
following Given \( X_{t-1} = x \) and \( Y_{t-1} = y \) Choose \( j \in
\set{1, 2, \dots, n} \) uniformly at random. Replace \( x_i \) and \( y_i
\) with the \emph{same} random bit chosen as the value of a fair
Bernoulli random value.  For example, on \( 10 \)-dimensional hypercube
suppose the \( X \) random walk is at \( 0011010011 \) and the \( Y \)
random walk is at \( 0110001010 \).  Position \( 6 \) is selected at
random, and the Bernoulli random variable comes up \( 1 \).  The \( 6 \)th
coordinate is \( 1 \), leaving \( X \) unchanged, but the \( Y \) chain
moves to \( 0110011010 \).  This example makes it obvious each of the
walks constructed in this way is a lazy random walk on the hypercube.

If \( \tau \) is the first time when all the coordinates have been
selected at least once, then the, then the two walks agree with each
other from time \( \tau \) onward.  If the two initial states agree in
some coordinates, the first time the walks coincide could be strictly
before \( \tau \).  Let \( R_t \) be the number of unselected
coordinates at time \( t \).  The random variable decreases rapidly at
first, then it slows down as it becomes harder to select an unused
coordinates.  This is basically the coupon collectors problem, so \( \E{R_t}
= n(1 - 1/n)^t \).

\begin{remark}
    The usual formulation of the coupon collectors problem%
    \index{coupon
    collectors problem}
    has \( n \) different types of coupons or prizes in a cereal box.
    On each draw, one gets a coupon or prize equally likely to be any
    one of the \( n \) types.  The goal is to find the expected number
    of coupons one needs to gather before getting a complete set of at
    least one of each type.
\end{remark}

\begin{theorem}
    \begin{enumerate}
        \item
            Let \( X_t, Y_t \) be coupled Markov chains with \( X_0 = x \)
            and \( Y_0 = y \).
        \item
            Let \( \Probsub{x,y}{\cdot} \) be the probability
            distribution for \( X_t, Y_t \).
        \item
            Let \( \tau_{\text{coal}} \) be the coalescence time of the
            chains, that is
            \[
                \tau_{\text{coal}} = \min \setof{t}{X_s = Y_s, s \ge t}.
            \]
        \item
            Then
            \[
                \| P^t(x, \cdot) - P^t(y,\cdot)\|_{TV} \le \Probsub{x,y}
                {\tau_{\text{coal}} > t}.
            \]
    \end{enumerate}
\end{theorem}

\begin{proof}
    \begin{enumerate}
        \item
            The first observation is that \( P^t(x, z) = \Probsub{x,y}{X_t
            = z} \) and \( P^t(y, z) = \Probsub{x,y}{Y_t = z} \).
        \item
            Using the bound on the total variation metric observed in~\eqref
            {eqn:convergencestationary:tvbound},
            \[
                \| P^t(x, \cdot) - P^t(y,\cdot)\|_{TV} \le \Probsub{x,y}
                {X_t \ne Y_t}.
            \]
        \item
            But \( \Probsub{x,y}{X_t \ne Y_t} = \Probsub{x,y}{\tau_{\text
            {coal}} > t} \) which completes the proof.
    \end{enumerate}
\end{proof}

Two convenient notations are
\[
    d(t) = \max_{x \in \mathcal{X}} \| P^t(x, \cdot) - \pi \|_{TV}
\] and
\[
    \bar{d}(t) = \max_{x \in \mathcal{X}} \| P^t(x, \cdot) - P^t(y,\cdot)\|_
    {TV}.
\]

\begin{lemma}
    \[
        d(t) \le \bar{d}(t) \le 2 d(t)
    \]
\end{lemma}

\begin{proof}
    \begin{enumerate}
        \item
            For the first inequality, state with the simple property of
            stationarity \( \pi(z) = \sum_{y \in \mathcal{X}} \pi(y)
            \left[ P^t(y,z) \right] \).
        \item
            Now for any set \( A \) of states, sum the previous over the
            states \( z \in A \) so \( \pi(A) = \sum_{y \in \mathcal{X}}
            \pi(y) \left[ P^t(y,A) \right] \)
        \item
            Using the row sum probability for a Markov chain is \( 1 \)
            and applying absolute values
            \[
                \abs{P^t(x,A) - \pi(A)} = \abs{\sum_{y \in \mathcal{X}}
                \pi(y) \left[ P^t(x,A) - P^{t}(y,A)\right]}.
            \]
        \item
            By the triangle inequality, the definition of total
            variation as the maximum difference over all events and the
            definition of \( \bar{d}(t) \)
            \[
                \abs{\sum_{y \in \mathcal{X}} \pi(y) \left[ P^t(x,z) - P^
                {t(y,z)}\right]} \le \sum_{y \in \mathcal{X}} \pi(y) \|
                P^t(x,\cdot) - P^t(y,\cdot)\|_{TV} \le \bar{d}(t).
            \]
        \item
            Maximize the left side over \( X \) and \( A \) yields \( d(t)
            \le \bar{d}(t) \).
        \item
            For the second inequality, start with the triangle
            inequality for the total variation distance
            \[
                \| P^t(x, \cdot) - P^t(y,\cdot)\|_{TV} \le \| P^t(x,
                \cdot) - \pi\|_{TV} + \| \pi - P^t(y,\cdot)\|_{TV}.
            \]
        \item
            Then
            \[
                \bar{d}(t) = \max_{x \in \mathcal{X}} \| P^t(x, \cdot) -
                P^t(y,\cdot)\|_{TV} \le \max_{x \in \mathcal{X}}\| P^t(x,
                \cdot) - \pi\|_{TV} + \max_{y \in \mathcal{X}} \| \pi -
                P^t(y,\cdot)\|_{TV} = 2 d(t).
            \]
    \end{enumerate}
\end{proof}

\begin{lemma}
    Sample uniformly with replacement from the set \( \set{1, 2, \dots n}
    \).  Let \( \tau \) be the number of draws required until each
    element has been drawn at least once.  Then
    \[
        \Prob{\tau > n \log n + c n} \le \EulerE^{-c}
    \] for \( c \ge 0 \) and \( n \ge 1 \).
\end{lemma}

\begin{proof}
    Let \( m = n \log n + c n \).  For each integer \( b \) let \( A_b \)
    be the event ``integer \( b \) not drawn in the first \( m \) draws.
    Then
    \[
        \Prob{ \tau > m} = \Prob{ \bigcup_{b=1}^n A_b } \le \sum_{b=1}^n
        \Prob{A_b} = n \left( 1 - \frac{1}{n} \right)^m \le n \EulerE^{-m/n}
        = \EulerE^ {-c}.
    \] See the exercises for a proof of the second inequality.
\end{proof}

\begin{corollary}
    \[
        t_{\text{mix}} \le n \log n + \log(1/\epsilon) n.
    \]
\end{corollary}

Given \( R_t \), the Markov chains \( X_t \) and \( Y_t \) have Hamming
weights that are binomial random variables,
\begin{align*}
    \operatorname{wt}
    (X_t) &=
    \operatorname{Bin}
    (n - R_t, 1/2) \\
    \operatorname{wt}
    (X_t) &=
    \operatorname{Bin}
    (n - R_t, 1/2) + R_t.
\end{align*}

Therefore, at time \( t \approx \frac{n}{2} \log n \)
\begin{align*}
    \E{%
    \operatorname{ht}
    (X_t)} &\approx \frac{n}{2} - \frac{\sqrt{n}}{2}, \\
    \E{%
    \operatorname{ht}
    (Y_t)} &\approx \frac{n}{2} + \frac{\sqrt{n}}{2}.
\end{align*}
and
\begin{align*}
    \sigma{%
    \operatorname{ht}
    (X_t)} &= O(\sqrt{n}) \\
    \sigma{%
    \operatorname{ht}
    (Y_t)} &= O(\sqrt{n}),
\end{align*}
where \( \sigma(X) \) is the standard deviation of a random variable.

\begin{theorem}
    Let \( \epsilon > 0 \).  Then there exists an \( \alpha \ge 0 \) so
    that for \( t(\alpha) = \frac{n}{2} \log n + \alpha n \), \( \Prob{\tau
    > t(\alpha)} < \epsilon \).
\end{theorem}

\begin{lemma}
    \label{thm:convergence:lem713}%
    \index{coupon collectors problem}
    Consider the coupon collecting problem with \( n \) distinct coupon
    types, and let \( I_j(t) \) be the indicator of the event that the \(
    j \)-th coupon has not been collected by time \( t \).  Let \( R_t =
    \sum_{\nu = 1}^n I_j(t) \) be the number of coupon types not
    collected by time \( t \). The random variables \( I_j(t) \) are
    negatively correlated, and letting \( p =(1 - 1 n)^t \), we have for
    \( t \ge 0 \),
    \begin{align*}
        \E(R_t) &= n p \\
        \Var{R_t} &= n p (1- p) \le \frac{n}{4}.
    \end{align*}
\end{lemma}

\begin{remark}
    With this notation, notice the similarity to the mean and variance
    of a binomial random variable.
\end{remark}

\begin{proof}
    \begin{enumerate}
        \item
            Since \( I_j(t) = 1 \) if and only if the first \( t \)
            coupons are not of type \( j \), immediately
            \begin{align*}
                \E(I_j(t) ) &= \left( 1 - \frac{1}{n} \right)^n = p \\
                \Var{I_j(t)} &= p (1- p).
            \end{align*}
        \item
            Similarly, for \( j \ne k \),
            \[
                \E{ I_j(t) I_k(t) } = \left( 1 - \frac{2}{n} \right)^t.
            \]
        \item
            Therefore
            \[
                \Cov{I_j(t) I_k(t)} = \left( 1 - \frac{2}{n} \right)^t -
                \left( 1 - \frac{1}{n} \right)^{2t} \le 0.
            \]
    \end{enumerate}
\end{proof}

\begin{definition}
    When \( \mu \) is a probability distribution on \( \mathcal{X} \),
    and \( f : \mathcal{X} \to \Lambda \) is a function on \( \mathcal{X}
    \), write \( \mu f^{-1} \) for the probability distribution defined
    by
    \[
        (\mu f^{-1})(A) = \mu(f^{-1}(A))
    \] for \( A \subset \Lambda \).  When \( X \) is an \( \mathcal{X} \)-valued
    random variable with distribution \( m \), then \( f(X) \) has
    distribution \( \mu f^{-1} \) on \( \Lambda \).
\end{definition}

\begin{lemma}
    \label{thm:convergencestationary:lem710}
    \begin{enumerate}
        \item
            Let \( \mu \) and \( \nu \) be probability distributions on \(
            \mathcal{X} \).
        \item
            Let \( f:X \to \Lambda \) be a function on \( \mathcal{X} \)
            where \( \Lambda \) is a finite set.
    \end{enumerate}
    Then \( \| \mu - \nu \|_{TV} \ge \| \mu f^{-1} - \nu f^{-1} \|_{TV} \).
\end{lemma}

\begin{proof}
    \begin{enumerate}
        \item
            First (read the parentheses carefully)
            \[
                \abs{\mu f^{-1}(B) - \nu f^{-1}(B)} = \abs{\mu (f^{-1}(B))
                - \nu (f^{-1}(B))}.
            \]
        \item
            Then
            \[
                \max_{B \subset \Lambda}\abs{\mu f^{-1}(B) - \nu f^{-1}(B)}
                \le \max_{A \subset \mathcal{X}}\abs{\mu (A) - \nu (A)}.
            \]

    \end{enumerate}
\end{proof}

\begin{remark}
    Use this lemma to lower bound the distance of some chain from
    stationarity in terms of the corresponding distance for a projection
    or lumping of that chain.  To do so, take \( \Lambda \) to be the
    relevant partition of X.
\end{remark}

\begin{proposition}
    \label{thm:convergencestationary:prop79}
    \begin{enumerate}
        \item
            Let \( f:\mathcal{X} \to \Reals \).
        \item
            Let \( \mu \), \( \nu \) be two probability distributions on
            \( \mathcal{X} \).
        \item
          Define \( \sigma_{\star}^2 = \max[ \Varsub{\mu}{f},
            \Varsub{\nu}{f} ] \).
        \item
            Suppose \( \abs{ \Esub{\mu}{f} - \Esub{\nu}(f)} \ge r \sigma_
            {\star} \).
    \end{enumerate}
    Then
    \[
        \| \mu - \nu \|_{TV} \ge 1 - \frac{8}{r^2}.
    \]
\end{proposition}

\begin{proof}
    \begin{enumerate}
        \item
            Suppose without loss of generality that \( \Esub{\mu}{f} \le
            \Esub{\nu}{f} \).
        \item
            Let \( A = (\Esub{\mu}{f} + r\sigma_{\star}/2, \infty) \),
            then Chebyshev's Inequality yields \( \mu f^{-1}(A) \le
            \frac{4}{r^2} \) and \( \nu f^{-1}(A) \ge 1 - \frac{4}{r^2} \).
        \item
            Then
            \[
                \| \mu f^{-1} -\nu f^{-1} \|_{TV} \ge 1 - \frac{8}{r^2}.
            \]
        \item
            Then apply Lemma~%
            \ref{thm:convergencestationary:lem710} to complete the
            proof.
    \end{enumerate}
\end{proof}

\begin{corollary}
    \label{thm:convergencestationary:cor79} Suppose For a Markov chain \(
    X_t \) with transition probability matrix \( P \), the function \( f
    \) satisfies
    \[
        \abs{ \Esub{x}{f(X_t)} - \Esub{\pi}{f}} \ge r \sigma_{\star}
    \] then
    \[
        \| P^t(x, \cdot) - \pi \|_{TV} \ge 1 - \frac{8}{r^2}
    \]
\end{corollary}

\begin{proof}
    \begin{enumerate}
        \item
            See the exercises.
    \end{enumerate}
\end{proof}

\begin{proposition}
    For the lazy random walk on the \( n \)-dimensional hypercube
    \[
        d \left( \frac{1}{2} n \log n - \alpha n \right) \ge 1 - 8^{2 -
        2\alpha}.
    \]
\end{proposition}

\begin{proof}
    \begin{enumerate}
        \item
            Apply Proposition~%
            \ref{thm:convergencestatstionary:prop79} with \( f =
            \operatorname{wt}
            \).  The walker started \( \bar{1} \) is \( X_t \).
        \item
            As \( \pi \) is uniform on \( {0,1}^n \), the distribution
            of the random variable \( W \) is \(
            \operatorname{Bin}
            (n,1/2) \), so \( \Esub{\pi}{W} = n/2 \) and \( \Varsub{x}{W}
            = n/4 \).
        \item
            Recall that \( R_t \) is the number of coordinates not
            updated by time \( t \).  When starting from \( \bar{1} \),
            the conditional distribution of \(
            \operatorname{wt}
            (X_t) \), given \( R_t = r \) is \( r +
            \operatorname{Bin}
            (n-r, 1/2) \).
        \item
            Consequently,
            \[
                \Esub{1}{%
                \operatorname{wt}
                (X_{t}) \given R_t} = R_t + \frac{(n-R_t)}{2} = \frac{1}
                {2}(R_t + n).
            \]
        \item
            Using Lemma~%
            \ref{thm:convergence:lem713}
            \[
                \Esub{\bar{1}}{%
                \operatorname{wt}
                (X_t)} = \frac{n}{2} \left[ 1 + \left( 1- \frac{1}{n}
                \right)^t \right].
            \]
        \item
            Recall the identity for variance using conditionals, applied
            here as
            \begin{align*}
                \Varsub{\bar{1}}{%
                \operatorname{wt}
                (X_t)} &= \Varsub{\bar{1}}{\E{%
                \operatorname{wt}
                (X_t)} \given R_t} + \Esub{\bar{1}}{\Varsub{\bar{1}}{%
                \operatorname{wt}
                (X_t)} | R_t} \\
                &= \frac{1}{4} \Varsub{\bar{1}}{%
                \operatorname{wt}
                (X_t)} + \frac{1}{4} \left[ n - \Esub{\bar{1}}{R_t}
                \right].
            \end{align*}
        \item
            Again using Lemma~%
            \ref{thm:convergence:lem713}, \( R_t \) is the sum of
            negatively correlated indicator functions and consequently, \(
            \Varsub{\bar{1}}{R_t} \le \Esub{\bar{1}}{R_t} \) so \(
            \Varsub{\bar{1}}{R_t} \le n/4 \).
        \item
            Set
            \[
                \sigma = \sqrt{\max[\Varsub{\pi}{W}, \Varsub{\bar{1}}{%
                \operatorname{wt}
                (X_t)}]} = \frac{\sqrt{n}}{2}.
            \]
        \item
            Then
            \[
                \abs{\Esub{\pi}{W} - \Esub{\bar{1}}{%
                \operatorname{wt}
                (X_t)} } = \frac{n}{2} \left( 1 - \frac{1}{n} \right)^t
                = \sigma \sqrt{n} \left( 1 - \frac{1}{n} \right)^t.
            \]
        \item
            Set \( t_n = \frac{1}{2}(n-1)\log n - (\alpha-1)n > \frac{1}
            {2}n \log n - \alpha n \) and using \( (1-1/n)^{n-1} >
            \EulerE^{-1} > (1 - 1/n)^n \),
            \[
                \abs{\Esub{\pi}{W} - \Esub{\bar{1}}{%
                \operatorname{wt}
                (X_t)} } \EulerE^{\alpha-1} \sigma.
            \]
        \item
            Applying Lemma~%
            \ref{thm:convergence:prop79} gives
            \[
                d\left(\frac{1}{2} n \log n - \alpha n \right) \ge \| P^
                {t_n}(\bar{1}, \cdot) - \pi \|_{TV} \ge 1 - 8\EulerE^{2-2\alpha}.
            \]
    \end{enumerate}
\end{proof}

\subsection*{Continuous State Space and Point Process MCMC}

Consider three particles each randomly located in the square \( [0,1]^2
\subset \Reals^2 \) with positions \( (x_{i1}, x_{i2}) \) for \( i =
1,2,3 \).  The state space is \( \mathcal{X} = [0,1]^6 \).  Suppose the
positions of the particles are distributed according to the density
\[
    \pi(x) = \pi(x_1,x_2,x_3) = \frac{1}{z} \exp\left[ -C \sum\limits_{i=1}^3
    \|x_i\| - \sum\limits_{i<j} \frac{1}{\| x_i - x_j \|} \right]
\] where \( \| \cdot \| \) is the usual Euclidean norm on \( \Reals^2 \),
\( C \) and \( D \) are fixed positive constants and \( z \) is the
normalizing constant or partition function.  In this density the first
sum pushes the particle towards the origin, the second sum pushes them
away from each other.

Using the Metropolis algorithm, create a Markov chain with \( \pi \) as
its stationary distribution.  Given \( X_n \), first choose \( y \in [0,1]^6
\) from the uniform distribution on \( \mathcal{X} \).  Then with
probability \( r = \min{1, \pi(y)/\pi(x)} \), accept \( X_{n+1} = y \),
otherwise reject \( y \) and \( X_{n+1} = x \).  Then this Markov chain
has \( \pi \) as its stationary distribution.

\begin{lemma}
    The Metropolis algorithm Markov chain for the \( 3 \) particle
    system with stationary distribution
    \[
        \pi(x) = \pi(x_1,x_2,x_3) = \frac{1}{z} \exp\left[ -C \sum\limits_
        {i=1}^3 \|x_i\| - \sum\limits_{i<j} \frac{1}{\| x_i - x_j \|}
        \right]
    \] satisfies a uniform minorization condition with \( n_0 = 1 \) and
    \( \epsilon = (0.488) \exp(-C(3 \sqrt{2}) - D(12 - 3/ \sqrt{2})) \)
\end{lemma}

\begin{remark}

\end{remark}
With \( 3 \sqrt{2} \approx 4.243 \) and \( 12 - 3/\sqrt{2} \approx 9.879
\), the estimate \( \epsilon = (0.488) \exp(-C(4.25) - D(9.88)) \) is
simpler.

\begin{proof}
    \begin{enumerate}
        \item
            To avoid configurations where the particles are close
            together, creating a near division by \( 0 \), set
            \[
                \mathcal{X}' = \setof{(x_1,x_2,x_3)}{\| x_i - x_j \| \ge
                1/4}.
            \]
        \item
            Being closed and bounded, the minimum value
            \[
                m = \min_{x,y \in \mathcal{X}} \frac{\pi(y)}{\pi(x)} > 0
            \] is assumed on \( \mathcal{X} \).
        \item
            Let \( A \subseteq \mathcal{X}' \).  Then from any state \(
            x \in \mathcal{X} \), the chain will move into \( A \) on
            the next step provided the proposed new configuration \( y \)
            is in \( A \) and the proposed configuration is accepted
            according to the Bernoulli random variable.
        \item
            Hence
            \begin{align*}
                P(x,A) &= \int_A P(x,\df{y}) \\
                &= \int_A \min{1, \frac{\pi(y)}{\pi(x)}} \df{y} \\
                &\ge \int_{A \intersect \mathcal{X}'} m \df{y} \\
                & m
                \operatorname{Leb}
                (A \intersect \mathcal{X}')
            \end{align*}
            where \(
            \operatorname{Leb}
            (\cdot) \) is Lebesgue measure on \( \Reals^6 \).
        \item
            Set \( \epsilon =
            \operatorname{Leb}
            (\mathcal{X}') \) and \( \mu(A) =
            \operatorname{Leb}
            (A \intersect \mathcal{X}')/
            \operatorname{Leb}
            (\mathcal{X}') \), then \( \epsilon > 0 \) and \( \mu \) is
            a probability measure and the uniform minorization condition
            is satisfied.
        \item
            Numerical convergence bounds require estimation of \(
            \operatorname{Leb}
            (\mathcal{X}') \) and \( m \).
        \item
            In order for \( (x_1,x_2,x_3) \in \mathcal{X}' \), first
            choose \( x_1 \in [0,1]^2 \) with area \( 1 \).  Then choose
            any \( x_2 \in [0,1]^2 \setminus B(x_1, 1/4) \) with area
            greater than \( 1 - (\text{pi})(1/4)^2 \).  Finally choose
            any \( x_3 \in [0,1]^2 \setminus (B(x_1, 1/4) \union B(x_2,
            1/4)) \) with area \( 1 - 2 \cdot (\text{pi})(1/4)^2 \).  (Here
            \( B(x,r) \) is the ball in \( \Reals^2 \) with center \( x \)
            and radius \( r \) and \( (\text{pi}) \) is used instead of
            the traditional symbol to avoid confusion with the
            stationary distribution.  This is the only place so far in
            these notes where the notations have collided!)
        \item
            Hence \(
            \operatorname{Leb}
            (\mathcal{X}') \ge 1 (1 - 3.14/16)(1 - 3.14/8) \ge 0.488 \).
        \item
            For any \( x_{i}, x_j \in \mathcal{X}' \), \( 0 \le \|x\|
            \le \sqrt{2} \) and \( 1/4 \le \| x_i x_j\| \le \sqrt{2} \).
            Then \( 0 \le \sum\limits_{i=1}^3 \|x_i\|\le 3 \sqrt{2} \)
            and
            \[
                \frac{3}{\sqrt{2}} \le \sum\limits_{i<j} \frac{1}{\| x_i
                - x_j \|} \le 12.
            \] Then
            \[
                m \ge \frac{\EulerE^{-C \cdot (3 \sqrt{2}) - D(12)}}{\EulerE^
                {-C \cdot (0) - D(3/\sqrt{2})}} = \EulerE^{-C \cdot (3
                \sqrt{2}) - D(12- 3/\sqrt{2})}.
            \]
    \end{enumerate}
\end{proof}

\begin{remark}
    The proof can be modified to give slightly better estimates by
    taking the removed balls to have smaller radii.
\end{remark}

If \( C = D = \frac{1}{10} \), then \( \epsilon = 0.117 \) giving the
convergence bound
\[
    \|
    \operatorname{dist}
    (X_n) - \pi \|_{TV} \le (0.883)^n.
\] After \( 38 \) steps, the total variation distance between the Markov
chain and the stationary distribution is less than \( 0.01 \).

\subsection*{Pseudo-Minorization Conditions}

The coupling construction in the Minorization Estimate was a pairwise
construction, it only considered states \( x \) and \( y \) at a time.
Instead, replace \( \mu(\cdot) \) with \( \mu_{x,y}(\cdot) \) allowing
it to depend on \( x \) and \( y \).  Then \( C \) is called a
pseudo-small set and the Minorization Estimate Theorem continues to
hold.  See the exercises for the proof.  On a finite state space, choose
\[
    \epsilon = \min_{i,j \in \mathcal{X}}\sum_{z \in \mathcal{X}} \min_{i,j
    \in \mathcal{X}} \left[ (P^{n_0})_{i,z} (P^{n_0})_{j,z} \right] > 0
\] with
\[
    \mu_{i,j}(z) = \frac{\min_{i,j \in \mathcal{X}} \left[ (P^{n_0})_{i,z}
    (P^{n_0})_{j,z} \right]}{\sum_{\nu \in \mathcal{X}} \min_{i,j \in
    \mathcal{X}} \left[ (P^{n_0})_{i,z} (P^{n_0})_{j,z} \right] }.
\] Then the chain will satisfy an \( n_0 \)-minorization condition, for
all \( i,j,z \in \mathcal{X} \), \( (P^{n_0})_{i,z} \ge \epsilon \mu_{i,j}
(z) \) and \( (P^{n_0})_{j,z} \ge \epsilon \mu_{i,j}(z) \).  Then again
\[
    \|
    \operatorname{dist}
    (X_n) - \pi \|_{TV} \le (1-\epsilon)^{\lfloor n/n_0 \rfloor}.
\]

\begin{example}
    Consider the random walk on the \( 3 \times 3 \) square lattice.  In
    \( P^2 \), the minimum values of \( \sum_{z \in \mathcal{X}} \min_{i,j
    \in \mathcal{X}} \left[ (P^{n_0})_{i,z} (P^{n_0})_{j,z} \right] \)
    occur at \( (i,j) = (3,7) \) and \( (1,9) \), corresponding to
    opposite corners of the square lattice.  Then calculating (see the
    exercises) \( \epsilon = \frac{1}{3} \).  Therefore
    \[
        \|
        \operatorname{dist}
        (X_n) - \pi \|_{TV} \le (1-\frac{1}{3})^{\lfloor n/2 \rfloor} =
        \left( \frac{2}{3} \right)^{\lfloor n/2 \rfloor}
    \] To get the distribution of the random walk within \( 0.01 \) of
    the stationary distribution requires \( 24 \) steps.  Compare this
    to the estimates of \( 78 \) steps with the Minorization Condition,
    and \( 6 \) step using the eigenvalues of \( P \).
\end{example}

\subsection*{Unbounded State Spaces}

The uniform minorization condition gives a good quantitative convergence
bound.  However, often on unbounded state spaces, the minorization
condition cannot be satisfied uniformly, only on some subset \( C
\subset \mathcal{X} \).  In such cases, adjust the previous \( n_0 = 1 \)
coupling construction, as follows.  First choose \( X0 \sim \mu(\cdot) \)
and \( X_0' \sim \pi(\cdot) \) independently, and then inductively for \(
n = 0, 1, 2, \dots \):
\begin{enumerate}
    \item
        If \( X_n = X_n' \), choose \( X_{n+1} = X_{n+1}' \sim P(X_n,
        \cdot) \).
    \item
        Else, if \( (X_n , X_n') \in C \times C \), flip a coin whose
        probability of Heads is \( \epsilon \), and then update \( X_{n+1}
        \) and \( X_{n+1}' \) in the same way as in step 2 of the
        previous uniform minorization construction.
    \item
        Else, if \( (X_n, X_n' ) \notin C \times C \), then
        conditionally independently choose \( X_{n+1} \sim P(X_n, \cdot)
        \) and \( X_{n+1}' \sim P(X_n', \cdot) \), i.e., the two chains
        are simply updated independently.
\end{enumerate}

The above construction provides good coupling bounds provided the two
chains return to \( C \times C \) often enough, but this last property
is difficult to guarantee.  Thus, to get convergence bounds, the Markov
chian needs a drift condition.  Basically, the drift condition
guarantees the chains will return to \( C \times C \) quickly enough to
still achieve a coupling.

\begin{definition}
    A Markov chain with a small set \( C \subseteq X \) satisfies a
    \defn{bivariate drift condition} if there exists a function
    \[
        h :  \mathcal{X} \times \mathcal{X} \to [1, \infty)
    \] and some \( \alpha > 1 \) such that \( \bar{P}h(x,y) \le h(x,y)/\alpha
    \) for \( (x,y) \notin C \times C \) where
    \[
        \bar{P}h(x,y)= \E{h(X_{n+1}, Y_{n+1}) \given X_n = x, Y_n = y}
    \] is the expected value of \( h(X_{n+1}, Y_{n+1}) \) on the next
    iteration, when the chains start from \( x \) and \( y \)
    respectively.
\end{definition}

Next define the quantity
\[
    B_{n_0} = \max{1, \alpha^{n_0}(1-\epsilon) \bar{R}h}
\] where
\[
    \bar{R}h(x,y) = \int_{\mathcal{X}} \int_{\mathcal{X}} (1-\epsilon)^2
    h(z,w) \left[ P^{n_0}(x, \df{z}) - \epsilon \mu(\df{z}) \right]
    \left[ P^{n_0}(y, \df{w}) - \epsilon \mu(\df{w}) \right].
\] The expression \( \bar{R}h(x,y) \) represents the expected value of \(
h(X_{n+{n_0}}, X_{n+{n_0}}) \) given \( X_n - x \)and \( X_n' = y \) and
the two chains fail to couple at time \( n \), meaning the coin comes up
Tails.

\begin{theorem}
    Consider a Markov chain on \( \mathcal{X} \), with \( X_0 = x \) and
    transition probabilities \( P \).  Suppose the minorization and
    bivariate drift conditions hold for some \( C \subseteq \mathcal{X} \),
    \( h :  \mathcal{X} \times \mathcal{X} \to [1, \infty] \),
    probability distribution \( \mu \), \( \alpha > 1 \), \( \epsilon >0
    \).  Then for any integers, \( 1 \le j \le n \) with \( B_{n_0} \)
    as above,
    \[
        \|
        \operatorname{dist}
        (X_n) - \pi \|_{TV} \le (1-\epsilon)^j + \alpha^{-n} B_{n_0}^{j-1}
        \Esub{Z \sim \pi}[h(x,Z)].
    \]

\end{theorem}

\begin{proof}[Sketch]
    \begin{enumerate}
        \item
            Create a second copy of the Markov chain with \( X_0' \sim
            \pi \) and use the coupling construction.
        \item
            Let \( N_n \) be the number of times the chain \( (X_n, X_n')
            \) is in \( C \times C \) by the \( n \)th step.
        \item
            Then by the coupling inequality.
            \begin{align*}
                \|
                \operatorname{dist}
                (X_n) - \pi \|_{TV} &\le \Prob{X \ne X_n'} \\
                &\le \Prob{X \ne X_n', N_{n-1} \ge j} | \Prob{X \ne X_n',
                N_{n-1} <j}.
            \end{align*}
        \item
            The first term suggests the chains have not coupled by time \(
            n \) despite visiting \( C \times C \) at least \( j \)
            times.  Since each such time gives them a chance of \(
            \epsilon \) to couple, the first term is less or equal to \(
            (1 - \epsilon)^j \).
        \item
            The second term is more complicated, but from the bivariate
            drift condition together with a martingale argument, it can
            be shown to be no greater than
            \[
                \alpha^{-n} B_{n_0}^{j-1} \Esub{Z \sim \pi}[h(x,Z)].
            \]
    \end{enumerate}
\end{proof}

Sometimes it could be hard to directly check the bivariate drift
condition so introduce the more easily-verified univariate drift
condition.  This gives a way to derive the bivariate condition from the
univariate one.

\begin{definition}
    A Markov chain with a small set \( C \) satisfies a \defn{univariate
    drift condition}%
    \index{univariate drift condition}
    if there are constants \( 0 < \lambda < 1 \) and \( b < \infty \)
    and a function \( V :  \mathcal{X} \to [1, \infty] \) such that
    \[
        PV(x) \le \lambda V(x) + b \indicator{C}(x)
    \] where \( PV(x) = \E{V(X_{n+1}) \given X_n = x} \).
\end{definition}

The univariate drift condition can be used to bound \( \Esub{\pi}{V} \)
in the following way.  Assuming \( \Esub{\pi}{V} < \infty \),
stationarity then implies \( \Esub{\pi}{V} \le \Esub{\pi}{V} + b \) so
then \( \Esub{\pi}{V} \le b/(1-\lambda) \).

\begin{proposition}
    Suppose:
    \begin{enumerate}
        \item
            The univariate drift condition is satisfied for some
            \begin{itemize}
                \item
                    \( V :  \mathcal{X} \to [1,\infty] \),
                \item
                    \( C \in \mathcal{X} \),
                \item
                    \( 0 < \lambda < 1 \),
                \item
                    \( b < \infty \).
            \end{itemize}
        \item
            \( d > ( b/(1-\lambda)) - 1 \).
    \end{enumerate}
    Then the bivariate drift condition is satisfied for the same \( C \)
    with \( h(x,y) = \frac{1}{2}[ V(x) + V(y)] \) and \( \alpha =
    \lambda + b/(d + 1) \).
\end{proposition}

\begin{proof}
    \begin{enumerate}
        \item
            Assume \( (x,y) \notin C \times C \).  Then either \( x
            \notin C \) or \( y \notin C \), so \( h(x,y) \ge (1 + d)/2 \).
        \item
            The univariate drift condition applied separately to \( x \)
            and \( y \) implies \( PV(x) + PV(y) \le \lambda V(x) +
            \lambda V(y) + b \).
        \item
            Therefore,
            \begin{align*}
                \bar{P}h(x,y) &= \frac{1}{2} \left[ PV(x) + PV(y) \right]
                \\
                &\le \frac{1}{2} \left[ \lambda V(x) + \lambda V(y) + b
                \right] \\
                &= \lambda h(x,y) + b/2 \\
                &\le \lambda h(x,y) + (b/2) \left[ h(x,y)/((1+d)/2)
                \right] \\
                &= \left[ \lambda + b/(1+d) \right] h(x,y).
            \end{align*}
    \end{enumerate}
\end{proof}

\begin{example}

    Let the state space be \( \mathcal{X} = \Reals \) with target
    density the double exponential or Laplace distribution.  \( \pi(x) =
    \EulerE^{-\abs{x}}/2 \).  It is easy to simulate the double
    exponential density using the standard technique of choosing a
    uniform random deviate and then applying the inverse of the
    cumulative distribution function, so this is an illustrative example
    rather than a practical simulation method for this random variate
    generator.

    Use another version of the Metropolis algorithm.  Choose the initial
    state \( x_0 \) uniformly at random from the interval \( [ -2, +2] \).
    At succeeding steps, first propose to move from state \( x \) to \(
    y \) chosen uniformly at random from the interval \( [x - 2, x + 2] \).
    With probability \( \min{1, \pi(y)/\pi(x)} \), accept the proposal \(
    y \) which becomes the new state, otherwise reject it and remain at \(
    x \).  Again, this procedure creates a Markov chain with \( \pi \)
    as its stationary distribution.

    To apply Theorem 2 requires minorization and drift conditions
    following from the next two lemmas.

    \begin{lemma}
        The Markov chain for sampling for the double exponential
        distribution satisfies a minorization condition with \( C = [-2,
        2] \), \( n_0 = 2 \), \( \epsilon = 1/(8 EulerE^2) \), and \(
        \mu(A) = \frac{1}{2}
        \operatorname{Leb}
        (A \intersect [-1,1]) \), where \(
        \operatorname{Leb}
        \) is Lebesgue measure on \( \Reals \).
    \end{lemma}

    \begin{proof}
        \begin{enumerate}
            \item
                Let \( x \in C \), Without loss of generality assume \(
                x \ge 0 \).
            \item
                First consider \( B \subset [-1,1] \) and let \( z \in [0,1]
                \) and \( y \in B \).
            \item
                Then \( [0,1] \subseteq [x-2, x+2] \) and \( B \subseteq
                [z-2, z+2] \).
            \item
                Hence the proposed density \( q \) satisfies \( q(z,z) =
                q(z,) = \frac{1}{2} \).
            \item
                Also, \( \pi(x) \le \EulerE^0 = 1 \) and \( \EulerE^{-1}
                \le \pi(y) \le 1 \) and \( \pi(z) \ge \EulerE^{-1} \),
                so if \( \alpha(x,z) = \min[1, \frac{\pi(z)}{\pi(x)}] \)
                is the probability of accepting a proposed move from \(
                x \) to \( z \), then \( \alpha(x,z) \ge \EulerE^{-1} \)
                and \( \alpha(z,x) \ge \EulerE^{-1} \).
            \item
                Then
                \begin{multline*}
                    P^2(x,B) \ge \int_B \int\limits_{x-2}^{x_2} q(x,z) q
                    (z,x) \alpha(z,y) \df{x} \df{y} \\
                    \ge \int_B \int_0^1 \frac{1}{4} \cdot \frac{1}{\EulerE}
                    \cdot \frac{1}{4} \cdot \frac{1}{\EulerE} \df{z} \df
                    {y} = \frac{1}{16 \EulerE^2}
                    \operatorname{Leb}
                    (B).
                \end{multline*}
            \item
                Finally for any \( A \subseteq \Reals \)
                \[
                    P^2(x,A) \ge P^2(x,A \intersect [-1,1]) \ge \frac{1}
                    {16 \EulerE^2}
                    \operatorname{Leb}
                    (A \intersect [-1,1]) = \frac{1}{8 \EulerE^2} \mu(A).
                \]
        \end{enumerate}
    \end{proof}

    \begin{lemma}
        The Markov chain for sampling for the double exponential
        distribution satisfies a univariate drift condition with \( V(x)
        - \EulerE^{-\abs{x}/2} \), \( C = [-2,2] \), \( \lambda = 0.916 \),
        and \( b = 0.285 \).
    \end{lemma}

    \begin{proof}
        \begin{enumerate}
            \item
                Without loss of generality, assume \( x \ge 0 \).
            \item
                Note
                \[
                    PV(x) = \int\limits_{x-2}^{x+2} q(x,y) \abs{ V(y)
                    \alpha(x,y) + V(x)( 1 - \alpha(x,y))} \df{y}.
                \]
            \item
                First compute the integral on \( x \le y \le x + 2 \).
                Here \( \alpha(x,y) = \frac{\pi(y)}{\pi(x)} = \frac{\EulerE^
                {-y}}{\EulerE^{-x}} = \EulerE^{x-y} \) and \( q(x,y) =
                \frac{1}{4} \).
            \item
                \begin{align*}
                    &\int\limits_{x}^{x+2} q(x,y) \abs{V(y) \alpha(x,y)
                    + V(x)( 1 - \alpha(x,y))} \df{y} \\
                    &\qquad = \int\limits_x^{x+2} \frac{1}{4} \EulerE^{y/2}
                    \EulerE{x-y} \df{y} + \int\limits_x^{x+2} \frac{1}{4}
                    \EulerE^{x/2} (1 - \EulerE^{x-y}) \df{y} \\
                    &\qquad = \int\limits_x^{x+2} \EulerE^{-y/2} \df{y}
                    + \frac{1}{4} \EulerE^{x/2}\cdot 2 - \frac{1}{4}
                    \EulerE^{3x/2} \int\limits_x^{x+2} \EulerE^{-y} \df{y}
                    \\
                    &\qquad = \frac{1}{4} \EulerE^x \left[ -2 \EulerE^{-
                    (x+2)/2} + 2 \EulerE^{-x/2}\right] + \frac{1}{4}
                    \EulerE^{x/2}\cdot 2 - \frac{1}{4} \EulerE^{3x/2}
                    \left[ -\EulerE^{-x-2} + \EulerE^{-x} \right] \\
                    &\qquad = \frac{1}{4} \EulerE^{x/2} \left[ -2\EulerE^1
                    + 2 + 2 + \EulerE^{-2} - 1 \right] \\
                    &\qquad \frac{1}{4} \left[ 3 + \EulerE^{-2} - 2\EulerE^
                    {-1} \right] V(x) = \lambda_1 V(x)
                \end{align*}
                by setting \( \frac{1}{4} \left[ 3 + \EulerE^{-2} - 2\EulerE^
                {-1} \right] \approx 0.6 \).
            \item
                Case 1:  \( x \in (2,\infty) \) where \( (2,\infty) \not\subseteq
                [-2,2] \).  Then \( \alpha(x,y) = \min{1, \frac{\EulerE^
                {-abs{y}}}{\EulerE^{-\abs{x}}}} = 1 \) for all \( y \in
                [x-2,2] \) so
                \begin{multline*}
                    PV(x) = \int_{x-2}^x q(x,y) V(y) \df{y} + \lambda_1
                    V(x) = \\
                    \frac{1}{4} \int_{x-2}^x \EulerE^{y/2} \df{y} =\\
                    \frac{}{1/4} \EulerE^{x/2}\cdot 2 \cdot (1 - \EulerE^
                    {-1}) + \lambda_1 V(x) = (\frac{1}{2}(1-\EulerE^{-1})
                    + \lambda_1) V(x) \le 0.916 V(x)
                \end{multline*}
            \item
                Case 2:  \( x \in [1,2] \subseteq C \).  Again \( \alpha
                (x,y) = 1 \) for all \( y \in [x-2,x] \), so
                \begin{multline*}
                    PV(x) = \int_{x-2}^x q(x,y) V(y) \df{y} + \lambda_1
                    V(x) = \\
                    \frac{1}{4} \left( \int_{x-2}^x \EulerE^{-y/2} \df{y}
                    + \int_0^x \EulerE^{y/2} \df{y} \right) + \lambda_1
                    V(x) =\\
                    \frac{1}{4} \left( \int_{0}^{2-x} \EulerE^{y/2} \df{y}
                    + \int_0^x \EulerE^{y/2} \df{y} \right) + \lambda_1
                    V(x) =\\
                    \frac{1}{2} \left( (\EulerE^{x/2} + \EulerE^{1-x/2})
                    - 1 + \lambda_1 \EulerE^{x/2} \right).
                \end{multline*}
            \item
                Let \( z = \EulerE^{x/2} \).  Then numerically
                \[
                    \max_{x \in [1,2]} \left[ PV(x) - 0.916 V(x)\right]
                    = \max_{z\in [\sqrt{\EulerE}, \EulerE]} \left[ \frac
                    {1}{2} \left(z + \frac{\EulerE}{z}\right) - 1 +
                    \lambda_1 z -0.916 z \right] \le 0.13.
                \]
            \item
                Case 3:  \( x \in [0,1] \subseteq C \).  Then \( \alpha(x,y)
                = 1 \) for any \( y \in [-x,x] \).
                \begin{align*}
                    &\int\limits_{x-2}^{-x} \left[ q(x,y) \alpha(x,y) V(y)
                    + q(x,y) ( 1 - \alpha(x,y))V(x) \right] \df{y} \\
                    &\qquad \qquad + \int_x^x q(x,y) V(y) \df{y} +
                    \lambda_1 V(x) \\
                    &= \frac{1}{4} \EulerE^{x/2} \int_x^{2-x} \left(
                    \EulerE^{(x-y)/2} + 1 - \EulerE^{x-y} \right) \df{y}
                    + \frac{1}{2} \int_0^x \EulerE^{y/2} \df{y} +
                    \lambda_1 \EulerE^{x/2} \\
                    &= \frac{1}{4} \EulerE^{x/2} \left[ -2\EulerE^{x-1}
                    + \EulerE^{2(x-1)} - 2x + 3 \right] + \EulerE^{x/2}
                    - 1 + \lambda_1 \EulerE^{x/2}.
                \end{align*}
            \item
                Computing numerically,
                \[
                    \max_{x \in [0,1]}\left[ PV(x) - 0.916 V(x) \right]
                    \le 0.285.
                \]
            \item
                Combining these three cases and the symmetric versions
                for \( x < 0 \) shows that the univariate drift
                condition
                \[
                    PV(x) \le 0.916 V(x) + 0.285 \indicator{C}(x)
                \] holds for all \( x \in \mathcal{X} \).
        \end{enumerate}
    \end{proof}

    With these two lemmas apply the Proposition to derive a bivariate
    drift condition.  Here \( d = \inf_{x = C^C} V(x) = \EulerE \), and \(
    b/(1-\lambda) - 1 = 2.39 < \EulerE \).  So \( h(x,y) = \frac{1}{2}(V
    (x) + V(y)) \) satisfies a bivariate drift condition with \( \alpha^
    {-1} = \lambda + b/(d + 1) = 0.916 + 0.285/(\EulerE + 1) \approx
    0.993 \).

    To bound \( B_{n_0} = B_2 \), let \( D = [-6, 6] \).  Then \( P^2(x,D)
    = 1 \) for any \( x \in C \).  Thus
    \[
        \sup_{(x,y) \in C \times C} \bar{R} h(x,y) \le \sup_{(x,y) \in D
        \times D}h(x,y) = \sup_{x \in D} V(x) = \EulerE^3 < 20.1.
    \] So
    \[
        B_2 = \max{1, \alpha^2(1-\epsilon) \sup \bar{R} h} < (0.993)^{-2}
        \left(1 - \frac{1}{8 \EulerE^2} \right) (20.1) \approx 20.04.
    \]

    Let \( X_0 = 0 \), then
    \begin{multline*}
        \Esub{X \sim \pi}{h(0,Z)} = \Esub{X \sim \pi}{\frac{1}{2} \left(
        V(0) + V(Z) \right) }= \\
        \frac{1}{2} + \frac{1}{2} \frac{\int_{y \in \mathcal{X}} \EulerE^
        {\abs{y}/2 \EulerE^{-\abs{y}}} \df{y}}{\int_{y \in \mathcal{X}}
        \EulerE^{-\abs{y}}} \df{y} = \frac{1}{2} + \frac{1}{2} \cdot
        \frac{2}{1} = 2 (??)
    \end{multline*}
    An alternative estimation is instead to use the bound \( \Esub{X
    \sim \pi}{V} \le b/(1-\lambda) = 3.939 \), as above.  Therefore,
    using the Theorem
    \begin{align*}
        \|
        \operatorname{dist}
        (X_n) - \pi \|_{TV} &\le (1- \epsilon)^j + \alpha^{-n} B_2^{j-1}
        \Esub{X \sim \pi}{h(0,Z)} \\
        &\le (0.983)^{1 + n/439.56} + (2 \cdot (20.04)^{n/439.56})(0.993)^n.
    \end{align*}
    For example, setting \( n = 120{,}000 \) and \( j = 274 = 1+ n/439 \)
    the upper bound is
    \[
        \|
        \operatorname{dist}
        (X_n) - \pi \|_{TV} \le (0.983)^{1 + n/439.56} + (2 \cdot (20.04)^
        {n/439.56})(0.993)^n
    \] so that after \( 120{,}000 \) steps, the total variation distance
    between the Markov chain and the stationary distribution is less
    than \( 0.01 \).  The R script in the Algorithms section shows that
    after about \( 48{,}000 \) steps total variation distance between
    this Markov chain and the stationary distribution is less than \(
    0.05 \).  Either way, the convergence to a stationary distribution
    is much slower than other examples.

\end{example}
\visual{Section Starter Question}{../../../../CommonInformation/Lessons/question_mark.png}
\section*{Section Ending Answer}

\subsection*{Sources} This section is adapted from:

\nocite{}
\nocite{}

\hr

\visual{Algorithms, Scripts, Simulations}{../../../../CommonInformation/Lessons/computer.png}
\section*{Algorithms, Scripts, Simulations}

\subsection*{Algorithm}

\subsection*{Scripts}

% \input{ _scripts}

\hr

\visual{Problems to Work}{../../../../CommonInformation/Lessons/solveproblems.png}
\section*{Problems to Work for Understanding}
\renewcommand{\theexerciseseries}{}
\renewcommand{\theexercise}{\arabic{exercise}}

\begin{exercise}
    Prove the Minorization Condition Theorem under the less restrictive
    pseudo-minorization condition.
\end{exercise}
\begin{solution}
    To be done.
\end{solution}
% \begin{exercise}
%     \begin{enumerate}[label=(\alpha*)]
%     \item
% \end{enumerate}
% \end{exercise}
% \begin{solution}
%     \begin{enumerate}[label=(\alpha*)]
%     \item
% \end{enumerate}
% \end{solution}

\hr

\visual{Books}{../../../../CommonInformation/Lessons/books.png}
\section*{Reading Suggestion:}

\bibliography{../../../../CommonInformation/bibliography}

%   \begin{enumerate}
%     \item
%     \item
%     \item
%   \end{enumerate}

\hr

\visual{Links}{../../../../CommonInformation/Lessons/chainlink.png}
\section*{Outside Readings and Links:}
\begin{enumerate}
    \item
    \item
    \item
    \item
\end{enumerate}

\section*{\solutionsname} \loadSolutions

\hr

\mydisclaim \myfooter

Last modified:  \flastmod

\end{document}

%%% Local Variables:
%%% TeX-master: t
%%% End:
